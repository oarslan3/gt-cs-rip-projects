%%%%%%%%%%%%%%%%%%%%%%%%%%%%%%%%%%%%%%%%%%%%%%%%%%%%%%%%%%%%%%%%%%%%%%%%%%%%%%%%
%2345678901234567890123456789012345678901234567890123456789012345678901234567890
%        1         2         3         4         5         6         7         8

%\documentclass[letterpaper, 10 pt, conference]{ieeeconf}  % Comment this line out
                                                          % if you need a4paper
\documentclass[a4paper, 10pt, conference]{ieeeconf}      % Use this line for a4
                                                          % paper

\IEEEoverridecommandlockouts                              % This command is only
                                                          % needed if you want to
                                                          % use the \thanks command
\overrideIEEEmargins
% See the \addtolength command later in the file to balance the column lengths
% on the last page of the document



% The following packages can be found on http:\\www.ctan.org
%\usepackage{graphics} % for pdf, bitmapped graphics files
%\usepackage{epsfig} % for postscript graphics files
%\usepackage{mathptmx} % assumes new font selection scheme installed
%\usepackage{times} % assumes new font selection scheme installed
%\usepackage{amsmath} % assumes amsmath package installed
%\usepackage{amssymb}  % assumes amsmath package installed

\title{\LARGE \bf
Autonomous Offensive Passing Systems for Robot Soccer
}

%\author{ \parbox{3 in}{\centering Huibert Kwakernaak*
%         \thanks{*Use the $\backslash$thanks command to put information here}\\
%         Faculty of Electrical Engineering, Mathematics and Computer Science\\
%         University of Twente\\
%         7500 AE Enschede, The Netherlands\\
%         {\tt\small h.kwakernaak@autsubmit.com}}
%         \hspace*{ 0.5 in}
%         \parbox{3 in}{ \centering Pradeep Misra**
%         \thanks{**The footnote marks may be inserted manually}\\
%        Department of Electrical Engineering \\
%         Wright State University\\
%         Dayton, OH 45435, USA\\
%         {\tt\small pmisra@cs.wright.edu}}
%}

\author{Alex Cunningham and Philip Rogers% <-this % stops a space
\thanks{This work was not supported by any organization}% <-this % stops a space
% \thanks{H. Kwakernaak is with Faculty of Electrical Engineering, Mathematics and Computer Science,
%         University of Twente, 7500 AE Enschede, The Netherlands
%         {\tt\small h.kwakernaak@autsubmit.com}}%
% \thanks{P. Misra is with the Department of Electrical Engineering, Wright State University,
%         Dayton, OH 45435, USA
%         {\tt\small pmisra@cs.wright.edu}}%
}


\begin{document}



\maketitle
\thispagestyle{empty}
\pagestyle{empty}


%%%%%%%%%%%%%%%%%%%%%%%%%%%%%%%%%%%%%%%%%%%%%%%%%%%%%%%%%%%%%%%%%%%%%%%%%%%%%%%%
\begin{abstract}

This project adds new passing capabilities to the GT Robojackets RoboCup Small Size League (SSL) robot team. The existing system supports moving, shooting, and can play the rules of soccer, but does not currently handle multirobot activity, such as passing, in a reliable manner. We extend the current system to implement a robust offensive passing system where the robots will acquire ball control and then make a coordinated series of passes until it can shoot on goal. We demonstrate a multi-stage planner that creates initial plans with an analytical planner and then performs nonlinear constrained optimization over the positions of robots to create an improved plan that mixed low-level controllers can execute.  

\end{abstract}


%%%%%%%%%%%%%%%%%%%%%%%%%%%%%%%%%%%%%%%%%%%%%%%%%%%%%%%%%%%%%%%%%%%%%%%%%%%%%%%%
\section{INTRODUCTION}

We talk about the stuff here, with general description of the game, what makes it hard, what we will do.

\section{Related Work}
"STP: Skills, tactics and plays for multi-robot control in adversarial environments" Brett Browning, James Bruce, Michael Bowling, and Manuela Veloso, Journal of System and Control Engineering (2005). This paper is the primary technique used by top RoboCup teams for planning both locally (kicking) and globally (passing) in the soccer domain. We have a rudimentary implementation of the STP framework currently, but it does not support global planning at the level required for competitive play. One of the main goals of our project is to extend our existing implementation, as the STP paper describes, to support passing and generally improved global teamplay.

"Multi-robot team response to a multi-robot opponent team" James Bruce, Michael Bowling, Brett Browning, and Manuela Veloso, ICRA '03. This paper describes the CMDragons implementation of Skills, Tactics, and Plays (STP) framework, and how they use it to achieve teamwork plays such as passing. It presents an example of their playbook, which is also published in full online, and can be used as a starting point for our passing implementation. The CMDragons team uses extensive passing, including "One-touch" kicking and chip-kick passing.

"Real-Time Randomized Path Planning for Robot Navigation" James Bruce, Manuela Veloso, IROS '02. This paper describes a modification the RRT algorithm called "execution extended RRT" (ERRT) that improves the efficiency of the RRT algorithm by weighting the search-space and improving replanning. Both of these topics are directly applicable to the RoboCup domain, where adversarial players affect the search space, and replanning is constantly occurring. By using ERRTs, our goal is to be able to create more efficient plans at the local planning level, which will lead to improved global plans.

"Map-based Multiple Model Tracking of a Moving Object" Cody Kwok, Dieter Fox, Proceedings of RoboCup Symposium, 2004. Due to the camera position, it is common to lose track of the soccer ball when multiple robots are occluding the camera's view. This paper presents a Rao-Blackwellized Particle Filter (RBPF) for tracking of the ball. It uses a technique of using particle filters for the non-linear portion of the tracking, and an Extended Kalman Filter (EKF) for the linear portions. Examples of the non-linear ball dynamics are during kicking, bouncing, etc. We hope this will improve overall planning, and is a requirement for passing.


\section{METHODS}
Outline the system here

\subsection{Analytical Planning}
Describe the algorithm for the analytical planner here

\subsection{Optimization}
Describe how the optimization works, with some information on SAM\cite{Dellaert05rss} and optimization techniques.  

\section{EXPERIMENTS}
Describe the test scenarios, limitations, simulator.

\section{ANALYSIS}
In this section we talk about stuff with lots of math and stuff.  

\section{DISCUSSION}
We do more talking about stuff here, but more hand-wavy.


\bibliographystyle{unsrt}
\bibliography{refs}
% \begin{thebibliography}{99}
% 
% \bibitem{c1}
% J.G.F. Francis, The QR Transformation I, {\it Comput. J.}, vol. 4, 1961, pp 265-271.
% 
% \bibitem{c2}
% H. Kwakernaak and R. Sivan, {\it Modern Signals and Systems}, Prentice Hall, Englewood Cliffs, NJ; 1991.
% 
% \bibitem{c3}
% D. Boley and R. Maier, "A Parallel QR Algorithm for the Non-Symmetric Eigenvalue Algorithm", {\it in Third SIAM Conference on Applied Linear Algebra}, Madison, WI, 1988, pp. A20.
% 
% \end{thebibliography}

\end{document}
